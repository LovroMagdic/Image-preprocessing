\documentclass{article}
\usepackage[english]{babel}
\usepackage[letterpaper,top=2cm,bottom=2cm,left=3cm,right=3cm,marginparwidth=1.75cm]{geometry}
\usepackage{amsmath}
\usepackage{graphicx}
\usepackage[colorlinks=true, allcolors=blue]{hyperref}
\usepackage[utf8]{inputenc} % Required for inputting international characters
\usepackage[T1]{fontenc} % Output font encoding for international characters
\begin{document}
\begin{titlepage}
	\newcommand{\HRule}{\rule{\linewidth}{0.5mm}} % Defines a new command for horizontal lines, change thickness here
	
	\center % Centre everything on the page
	
	%------------------------------------------------
	%	Headings
	%------------------------------------------------
	
	\textsc{\LARGE Fakultet Elektrotehnike i Računarstva}\\[1.5cm] % Main heading such as the name of your university/college
	\vfill
	\textsc{\Large Projekt R}\\[0.5cm]
	\textsc{\large Tehnička dokumentacija}\\[0.5cm] % Minor heading such as course title
	\HRule\\[0.4cm]
	{\huge\bfseries Sustav za procesiranje dokumenata pisanih pisaćom mašinom}\\[0.4cm] % Title of your document
	
	\HRule\\[1.5cm]
	
	%------------------------------------------------
	%	Author(s)
	%------------------------------------------------
	
	\begin{minipage}{0.4\textwidth}
		\begin{flushleft}
			\large
			\text{Student}\\
			Lovro Magdić
		\end{flushleft}
	\end{minipage}
	~
	\begin{minipage}{0.4\textwidth}
		\begin{flushright}
			\large
			\text{Nastavnik}\\
			Juraj Petrović
		\end{flushright}
	\end{minipage}
        \vfill\vfill
	{\large Siječanj 12, 2023.} % Date, change the \today to a set date if you want to be precise
	\vfill % Push the date up 1/4 of the remaining page	
\end{titlepage}

\section{Opis razvijenog proizvoda}
Cilj ovog projekta bio je razviti sustav za procesiranje dokumenata pisanih mašinom za tipkanje. Sustav korisniku omogućuje lakše vizualno pregledavanje dokumenata, te također lakše računalno iščitavanje teksta na dokumentima. Rad sustava je koncipiran na način da sve što se od korisnika očekuje da valjana putanja gdje se dokumenti nalaze, sve ostalo sustav radi sam, odabir najboljih parametra za svaku sliku pojedinačno, rotacija dokumenata ako je potrebno te stvaranje mapa sa svakim korakom obrade dokumenta.\\
\\
Cijeli proces obrade se odvija se kroz naredbeni redak odnosno u razvojnom okruženju po odabiru korisnika. Za razvijanje sustava korišten je Visual Studio Code (VSCode).\\
\\
Sustav je razvijen za python3 te zahtjeva razvojnu biblioteku OpenCV. OpenCV je biblioteka programskih funkcija za računalni vid u stvarnom vremenu,  njezine funkcionalnosti koristimo za razne obrade, spremanja i čitanja dokumenata.

\section{Opis rada sustava}

\subsection{Binary threshold i Image deskewing}

U navedenom koraku koristimo navedene funkcije obrade: Binary threshold i Image deskew.\\
Threshold je segmentacijska funckija pomoću koje postižemo crno-bijelu sliku iz "grayscale" formata ili slike u boji.Preciznije za inicjalni threshold koristimo varijantu "Binary Threshold". Kao parametre prima granicu do koje nijanse sive postaju crne i iznad navedene granice nijanse sive postaju bijele. %ovdje umentni sliku Binary Threshold \\
Binary threshold nam omogućuje uklanjanje raznih artefakata koji bi otežali proces "Hough Line Transform".\\

Hough Line Transform je jednostvani transformacijski proces pomoću kojeg pronalazimo ravne linije na slici, u našem slučaju rubove papira na slici koja je prošla threshold. Nakon detekcije linija, računamo njihovu orijentaciju s obziorm na os apscise i time dobivamo kut zakrivljenosti. Navedeni kut odnosno kutevi nam olakšavaju postizanje uspravnosti papira na slici.\\ \\
%ovdje stavi primjer sa neke slike di su nacrtane linije
Određenim kutem zakrivljenosti ostaje nam samo rotacija slike. Za to koristimo definiranu funckiju "rotateImage" koja obavlja rotaciju slike pomoću integrirane fuckije "warpAffine" iz biblioteke openCV, koja rotira sliku s obzirom na centar slike. %tu ubaci sliku koda rotateImage
Bitno je za naglasiti da se rotacija ne provodi na slici koja je prošla threshold već na originalnoj slici, threshold slika nam služi samo za izračune.
%ubaci originalnu pored rotirane

\subsection{Contour detection}
Sljedeći korak u procesu obrađivanja slika je "contour detection". Izuzetno je bitan za kasniju primjenu "adaptive threshold" i za izuzimanje nepotrebnih informacija iz slike (crne pozadine i artefakata). Radi na jednostavnom principu označavanja kontinuiranih točaka te lokaliziranje istih u skupinu odnosno konturu.\\ \\
Detekciju kontura provodimo na threshold slici a konture primjenjujemo na originalnoj odnosno rotiranom originalu.\\ %ubaci sliku sa konturom na originalu
\\
Kako bi postigli optimalnu konturu za svaku sliku provodimo detekciju optimalnog parametra threshold funckije za savaku sliku zasebno.
Koristimo threshold u intervalu (100, 235) i računamo površinu konture koju je detektirao, određujemo razliku maksimalne konture i površine koju je sustav odredio za trenutni threshold. Optimalni threshold imat će najmanju razliku jer u tom slučaju imamo najprecizniju konturu stvarnom papiru na slici.

\newpage

\subsection{Gaussian blur}

Primjenom Gaussove funkcije u procesu zamućivanja slike smanjujemo detalje i šum prisutan na slici. Postignuti produkt nazivamo "Gaussian blur". %ubaci sliku gaussian blur

\subsection{Adaptive threshold}

Razina svjetlosti varira između slika čak i na jednoj slici možemo uočiti više razina svjetlosti to nam otežava određivanje optimalnog parametra thresholda. \\ Kako bi rješili taj problem koristimo funkciju "adaptive threshold" koja izračunava threshold parametar za piksele s obzirom na susjedne piksele. Time dobivamo različite vrijednosti thresholda za područja različitog osvjetljenja.

\subsection{Font thinning}

Kako bi olkšali optičko prepoznavanje znakova primjenjujemo proces stanjivanja fonta kojim također izbacujemo nepotreban šum oko znakova i postižemo detaljniju sliku. %ubaci primje stanjivanja i kod

\section{Upute za korištenje}
\section{Literatura}
\end{document}